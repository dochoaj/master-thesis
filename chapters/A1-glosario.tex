%--------Anexo 1
%--------Daniel Ochoa John
%--------13/10/2014

\chapter{Glosario}

\begin{itemize}
	\item \textbf{Colaborativo}: Proceso de interacción humana donde las personas realizan continuamente una dualidad de funciones entre cliente y realizador, para realizar un trabajo colectivo. El trabajo colaborativo puede ser explícito o implícito como ocurre en la Web 2.0.
	\item \textbf{Comunidad}: Conjunto de indivíduos que poseen intereses similares y que interactúan entre sí con mayor frecuencia que con otras comunidades.
	\item \textbf{Detección de comunidades}: Conjunto de técnicas que permiten explicitar las agrupaciones entre indivíduos basándose en el conjunto de interacciones llevadas a cabo por los integrantes de un contexto social. Es considerada parte de la teoría de grafos y muchas de sus aplicaciones en la social media tienen que ver con el \textit{clustering} y catalogación de grupos de interés.
	\item \textbf{Endpoint}: En el mundo de los servicios Web, un endpoint es la dirección absoluta en la que un servicio está disponible. Esta dirección debe ser conocida por el cliente del servicio y no necesariamente deben pertenecer al mismo sistema computacional. En el diseño de servicios moderno, la nomenclatura para los endpoints está regida por REST.
	\item \textbf{Filtrado basado en contenido}: Proporciona recomendaciones mediante la comparación de las preferencias de un usuario en el pasado, recomendando aquellos ítems que son similares.
	\item \textbf{Filtrado colaborativo}: Proporciona recomendaciones basándose en las valoraciones de otros usuarios, capturando la inteligencia colectiva que los usuarios tienen sobre un dominio en particular. Para realizar una recomendación, este tipo de sistemas pueden utilizar técnicas de User-User, calculando aquellos usuarios más parecidos al usuario activo y realizando recomendaciones en base a las preferencia de esos, o bien de Item-Item, en donde se utiliza un principio similar, pero buscando los ítems similares a las preferencias del ususario.
	\item \textbf{Folksonomía}: Manera orgánica y colaborativa de indexación social de contenido por medio de etiquetas simples en un espacio de nombres. Es una práctica explicitada en, por ejemplo, los hashtags de Twitter, en donde los usuarios comienzan a compartir contenido, que es similar, bajo una misma etiqueta. Como por ejemplo, #CHI para los partidos de Chile en el mundial de Brasil.
	\item \textbf{Inteligencia colectiva}: Término generalizado de la sociedad del conocimiento que tiene relación con la colaboración y concurso de muchos indivíduos en pos de un objetivo común. Este concepto ha sido impulsado con las nuevas tecnologías de la información, la Web 2.0 y los dispositivos inteligentes, todos podemos crear y refinar contenido de forma colaborativa.
	\item \textbf{Interacción}: En redes sociales, es el concepto referido a la manifestación en el intercambio de contenidos entre: dos usuarios, un usuario y una plataforma, un usuario y un contenido. Existen distintos tipos de interacciones basándose en si esta es explícita o implícita.
	\item \textbf{Interacción explícita}: Una interacción explícita es cuando un usuario interactúa de manera consciente con otro usuario, una plataforma o un contenido. Por ejemplo, si un usuario escribe en el muro personal de otro en una red social, es una interacción explícita.
	\item \textbf{Interacción implícita}: Una interacción implícita es cuando un usuario interactúa de manera indirecta con otro usuario, una plataforma o un contenido. Por ejemplo, si un usuario hace like en una publicación de otro, entonces está interactuando explícitamente con aquel contenido, pero también indirectamente con el usuario que publicó ese contenido.
	\item \textbf{Item}: Un item se refiere a un elemento individual que puede ser recomendado de acuerdo a un perfil de un usuario. Un item puede ser de cualquier tipo, como publicaciones, noticias, música, entre otros.
	\item \textbf{JSON}: JSON (Javascript Object Notation) es un estándar de representación e intercambio de datos entre aplicaciones. Típicamente es utilizado en la comunicación entre un cliente y un servicio a través de la web. 
	\item \textbf{Ontología}: Hace referencia a la formulación de un esquema conceptual en uno o varios dominios, con la finalidad de establecer y homologar la comunicación y el intercambio de información.
	\item \textbf{Red social}: Como concepto, es una forma de representación que puede darse a una estructura social, en donde dos o más indivíduos están relacionados de acuerdo a algún criterio (relación profesional, amistad, parentesco, entre otros). Como servicio, es una plataforma que facilita, fomenta y registra todo tipo de conexiones entre dos o mas indivíduos, facilitando la generación de contenido y la divulgación de contenido ya existente.
	\item \textbf{RESTful}: Hace relación a aquellos sistemas que están construídos bajo la filosofía REST (Representational State Transfer), que es una arquitectura de software para sistemas distribuídos que utilizan esquemas XML bajo el protocolo HTTP. Es ámpliamente utilizado en servicios que proveen acceso a información a otros sistemas.
	\item \textbf{Scraping}: Es una técnica utilizada mediante programas computacionales que permite extraer información de sitios web. Típicamente, estos programas imitan la navegación que un humano realizaría en los sitios web que están siendo analizados. Comúnmente son utilizados para monitorear y observar ciertos segmentos de la Web de manera recurrente.
	\item \textbf{Sistema de recomendación}: Sistemas computacionales que recomiendan objetos de información que son de interés para un usuario activo, basándose en preferencias explícitas o implícitas. Típicamente, los sistemas de recomendación predicen las preferencias de los usuarios recomendando items que probablemente le gustarán o serán de su interés. Existen diversas estrategias sobre las que se basan los principios de recomendación, como basarse en el contenido o bien, ser de tipo colaborativo.
	\item \textbf{Social media}: Plataformas de comunicación en línea donde el contenido se crea por los mismos usuarios. Estas plataformas son influyentes y masivas. De esta manera fomentan la interacción entre sus usuarios y los contenidos que son creados en ellas son masificados y viralizados. La diferencia de esto con un boca a boca tradicional, es que las interacciones son medibles y analizables por profesionales especializados, por lo que son una herramienta valiosa al momento de descubrir tendencias.
\end{itemize}